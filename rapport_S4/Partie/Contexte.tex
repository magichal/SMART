
\chapter{Contexte}

\section{Nature du besoin}

Les drones sont de plus en plus présents dans le monde moderne est font maintenant partie intégrante du paysage urbains. Il est en effet possible d'acheter pour 50\euro~  un drone miniature dans n'importe quel rayon de jouet de grandes surfaces, ... %Mais son usage ne s'arrête pas au loisir, de grandes firmes américaines comme Amazon souhaite utiliser ces drones pour livrer leur produit
Mais son usage ne s'arrête pas au loisir puisque l'actualité a montré que l'intrusion de drones dans des sites sécurisés représentaient un risque de sécurité majeur. Le risque de sécurité que représentent ces drones peut aussi s'étendre à d'autres lieux, moins sensibles, mais ou leur intrusion peut avoir des conséquences désastreuses comme un aérodrome de campagne ou au dessus d'un terrain de sport pendant une compétition.


Notre projet, SMART (System with Multi Antennas to Reorient a Target), doit répondre a ce problème en permettant de détecter ces drones. La solution envisagée est un système passif de goniométrie qui réceptionne les ondes émises par le drone puis utilise l'effet Doppler pour obtenir la direction d'émission par rapport à une antenne. Un dispositif muni de deux systèmes d'antennes sera alors en mesure d'obtenir la position approximative du drone a détecter. 




%%% Local Variables: 
%%% mode: latex
%%% TeX-master: "../rapport"
%%% End: 
