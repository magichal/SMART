\chapter{Tests Unitaires}


\begin{tabular}{|p{0.3\textwidth}  |m{0.3\textwidth}  |p{0.3\textwidth} |}	
\hline
Test & État & Commentaires \\ \hline
\multicolumn{3}{|c|}{PIC}\\ \hline
PIC16F628A & Échec & Réaction partielle du PIC aux entrées. Un autre protocole est nécéssaire \\ \hline
PIC12F675 & Échec & Réaction partielle du PIC aux entrées. Un autre protocole est nécéssaire \\ \hline
PIC18F4520 & Échec & Réaction partielle du PIC aux entrées. Un autre protocole est nécéssaire \\ 
\hline
\multicolumn{3}{|c|}{Plaques et pistes}\\ 
\hline
Test des pistes de la carte principale & Réussite & Test au multimètre. Passage courant OK, tension OK et faible résistance des pistes \\ 
\hline
Test des pistes de la carte d'affichage & Réussite & Test au multimètre. Passage courant OK, tension OK et faible résistance des pistes \\
\hline
\multicolumn{3}{|c|}{Filtre passe-bande}\\ 
\hline
Bande passante & Réussite & Amplitude Max à 2.45GHz, faible atténuation pour f>2.5GHz \\
\hline
Coefficient de réflexion & Réussite & S11 faible sur la bande passante \\
\hline
Coefficient de transmission & Réussite & log(S21) proche de 2dBm \\
\hline
\multicolumn{3}{|c|}{VCO}\\ \hline
Fréquence libre & Réussite & Mesure à 1.35GHz proche de la valeur constructeur \\ 
\hline
Tension efficace & Réussite & Tension idéale : 8V \\
\hline
Sortie VCO avec régulateur & Réussite & Fréquence de sortie 1.91GHz \\
\hline
\multicolumn{3}{|c|}{Down-converter}\\ \hline
Fréquence de sortie & Reporté & A reprogrammer \\
\hline
\end{tabular}

\begin{tabular}{|p{0.3\textwidth}  |m{0.3\textwidth}  |p{0.3\textwidth} |}	
\hline
\multicolumn{3}{|c|}{Raspberry}\\ \hline
Test d'allumage & Réussite & Fonctionne parfaitement sur alimentation secteur \\
\hline
Test Raspbian & Réussite & Installation et test des fonctionnalités python de Raspbian \\ 
\hline
Test GPIO & Réussite & Les ports GPIO choisis pour le projet fonctionnent dans les 2 sens \\
\hline
Test Client Raspberry & Réussite & Connection et communication avec le serveur (Ordinateur Michael) en local et en externe \\
\hline
\multicolumn{3}{|c|}{Application Android}\\ \hline
Lancement de l'application & Réussite & Testée sur 3 appareils différents \\
\hline
Communication avec le serveur & Réussite & Échange de messages simples dans les deux sens \\
\hline

\end{tabular}