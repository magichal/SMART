
\chapter{Contexte}

\section{Nature du besoin}

Les drones sont de plus en plus présents dans le monde moderne est font maintenant partie intégrante du paysage urbains. Il est en effet possible d'acheter pour 50\euro~  un drone miniature dans n'importe quel rayon de jouet de grandes surfaces comme Leclerc, Carefour, Géant Casino, ... %Mais son usage ne s'arrête pas au loisir, de grandes firmes américaines comme Amazon souhaite utiliser ces drones pour livrer leur produit
Mais son usage ne s'arrête pas au loisir puisque l'actualité a montré que l'intrusion de drones dans des sites sécurisés représentaient un risque de sécurité majeur.

Notre projet au nom de SMART (System with Multi Antennas to Reorient a Target) doit répondre a ce problème en proposant une solution de détection de drone. Pour cela nous allons utiliser un système passif qui réceptionne les ondes émises par le drone puis utilise l'effet Doppler sur ces ondes pour obtenir la direction d'émission. Ainsi grâce a deux dispositif il sera possible d'obtenir la position du drone a détecter. 




%%% Local Variables: 
%%% mode: latex
%%% TeX-master: "../rapport"
%%% End: 
