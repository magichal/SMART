\chapter{Validation Fonctionnel Usine}

\begin{tabular}[c]{|p{0.18\textwidth}|p{0.8\textwidth}|}
\hline
Date & 17 mai 2016 \\
Lieu & Ensta Bretagne \\
Nom & Beaudoin \\
Prénom & Maxime\\
\hline
Écart constaté & Il m'a été montré durant cette démonstration l'interface de SMART. J'ai pu constater l'absence des capteurs qui doivent détecter les drones. Cette absence a malheureusement été pénalisant pour la démonstration. Néanmoins l'idée est très intéressante et le travail qui a été effectué m'a convaincu quant à la réalisation du projet. J'ai également pu noter des points perfectibles, tel que:

\begin{itemize}
\item l'actualisation automatique dans l'application android,
\item la détection des altitudes des drones,
\item la détection de plusieurs drones,
\item l'affichage de plus d'information dans l'interface web.
\end{itemize}
 \\
\hline
Correction 

envisagée& \textit{\og L'actualisation automatique dans l'application android\fg{}}

Il est possible de réaliser une actualisation automatique, mais pour cela nous devrions utiliser le service google cloud messenger qui est payant. Pour des questions budgétaires nous avons choisi d'avoir pour le moment une actualisation manuelle.

\textit{\og La détection des altitudes\fg{}}

Nous avons envisagé de corriger ce problème en plaçant certains capteurs de manière horizontale. Ainsi nous aurions leur altitude.

\textit{\og Détection de plusieurs drones\fg{}}

Pour régler ce problème il nous faudrait repenser la méthode de calcul de la position du drones.

\textit{\og Afficher plusieurs informations dans l'interface\fg{}}

Nous avons déjà envisagé une amélioration dans ce sens, mais elle nous semblait pas essentiel dans le cadre de ce projet.

\\
\hline
\end{tabular}

%%% Local Variables: 
%%% mode: latex
%%% TeX-master: "../rapport"
%%% End: 