
\chapter{Documentation Technique à Raspbian}
\label{annexe:raspbian}

Raspbian (recommended for Raspberry Pi 1) – is maintained independently of the Foundation; based on the Debian ARM hard-float (armhf) architecture port originally designed for ARMv7 and later processors (with Jazelle RCT/ThumbEE and VFPv3), compiled for the more limited ARMv6 instruction set of the Raspberry Pi 1. A minimum size of 4 GB SD card is required for the Raspbian images provided by the Raspberry Pi Foundation. There is a Pi Store for exchanging programs.

    The Raspbian Server Edition is a stripped version with fewer software packages bundled as compared to the usual desktop computer oriented Raspbian.

    The Wayland display server protocol enables efficient use of the GPU for hardware accelerated GUI drawing functions.[104] On 16 April 2014, a GUI shell for Weston called Maynard was released.

    PiBang Linux – is derived from Raspbian.

    Raspbian for Robots – is a fork of Raspbian for robotics projects with Lego, Grove, and Arduino.

%%% Local Variables: 
%%% mode: latex
%%% TeX-master: "../rapport"
%%% End: 
