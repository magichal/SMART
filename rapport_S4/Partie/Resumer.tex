
\chapter*{Résumé}
\addcontentsline{toc}{chapter}{Résumé}




Ce document présente notre rapport sur le projet SMART \og System with Multi Antennas to Reorient a Target\fg{} encadré  par M. MANSOUR Ali. Ce projet comporte deux phases: la détection d'un drone puis sa neutralisation. Compte tenu du  temps imparti nous nous sommes concentrés sur la détection. De plus, nous avons choisi de réaliser notre détection avec de la radio-goniométrie, et de proposer à nos utilisateurs deux interfaces de contrôle: une application web, et une application Android.


A cause de retard dans la réception de certains composants nous n'avons pas pu concevoir le capteur qui permet de détecter les drones, mais nous avons implémenté tout le système de visualisation et d'alerte, et nous avons testé au maximum les composants qui nous ont été livrés. De cette manière nous pouvons nous assurer que l'arrivée des quelques composants manquants suffira pour terminer ce projet. ~\\

%Ce sujet du projet vise à détecter des drones et afficher leurs positions via une application Android appelée \og S.M.A.R.T Comm Center\fg{} ou une application Web, dans le but de lutter contre les intrusions causées par les drones sur des grands centres spécifiquement nucléaires. Ce concept permet de mettre en pratique plusieurs technologies. Les méthodes et algorithmes utilisée sont inspirés du Montréal 3v2. La particularité de ce projet est qu’il vise un taux de réussite de détection pouvant atteindre 99\%. Notre projet présente un intérêt croissant pour de nombreuses entreprises. De plus, il nous a permis d’appliquer nos connaissances en télécommunication et en informatique à un domaine pratique. Toutefois, si un tel projet devait être complété, à plus ou moins longue échéance,  il serait nécessaire de tester la validité des résultats obtenus. Ainsi, on a recours à faire plusieurs tests sur le fonctionnement de différents composantes utilisées.
%Le problème le plus gênant durant la réalisation du projet est le retard dans la réception des composantes commandés(les antennes radio-goniomètres et des microcontrôleurs).~\\

% Jusqu'ici, le projet  nous a permis:
% \begin{itemize}
% \item d’appliquer la méthode radio-goniométrique
% \item d’utiliser des Raspberry PI qui permettent de transmettre les données à un ordinateur  central
% \item afficher la localisation des drones à travers une interface graphique (ordinateur central)
% \item notifier le client via une application Android.
% \end{itemize}




Nous espérons que vous prendrez autant de plaisir à lire ce rapport que nous en avons pris durant tout le déroulement de ce projet. 


\vfill{}
\textbf{Mots clés:}

Détection, drone, application Android, radio-goniomètres, application web


%%% Local Variables: 
%%% mode: latex
%%% TeX-master: "../rapport"
%%% End: 
