\chapter{Validation Fonctions de Service}


\begin{tabular}{|p{0.2\textwidth}| |p{0.2\textwidth}  |m{0.22\textwidth}  |p{0.22\textwidth} |}	

     \hline
     Numéro & Désignation & État & Commentaires \\ \hline
     FS1 & Détecter des drones à portée de réception par les antennes dans un domaine de fréquence prédéfini & Non validé & L'absence de capteur fonctionnel ne permet pas de valider entièrement cette fonction \\ \hline
     FS2a & Retourner la position du drone à l'utilisateur en temps réel & Partiellement validé & La position est transmise en temps réel uniquement dans le cadre des simulations \\ \hline
     FS2b & Avoir une précision de l'ordre du mètre & Non validé & Capteur non fonctionnel, impossible de vérifier la précision exacte \\ \hline
     FS3 & Suivre les déplacements du drone en temps réel & Partiellement validé & Le système suit bien les drones simulés \\ \hline
     FS4a & Alerter l'utilisateur en cas de nouvelle détection par un message via un PC & Validé & Possible via l'interface web \\ \hline
     \hline
   \end{tabular}

\begin{tabular}{|p{0.2\textwidth}| |p{0.2\textwidth}  |m{0.22\textwidth}  |p{0.22\textwidth} |}	

	\hline
	Numéro & Désignation & État & Commentaires \\ \hline
   	FS4b & Alerter l'utilisateur en cas de nouvelle détection via l'application Android & Validé & Dans le cas ou le drone est détecté l'utilisateur est bien averti \\ \hline
   	FS5 & Analyser et retourner la vitesse de déplacement du drone & Non validé & Pas de capteur fonctionnel pour tester cette fonction \\ \hline
   	FS6 & Retourner la trajectoire du drone à l'utilisateur & Validé & Via l'interface web \\      
   	\hline
\end{tabular}
	
%%	\hline
%%	Numéro & Désignation & État & Commentaires  \hline
%%	FS1 & Détecter des drones à portée de réception par les antennes dans un domaine de fréquence prédéfini & Non validé & L'absence de capteur fonctionnel ne permet pas de valider entièrement cette fonction  \hline
	


%%% Local Variables: 
%%% mode: latex
%%% TeX-master: "../rapport"
%%% End: 