
\chapter{Difficultés rencontrées}
\label{chap:difficulte}

Durant ce semestre, de nombreux obstacles à la réalisation de notre projet ont été rencontré. Ces difficultés sont de sources et de natures différentes, et ont conduit à une évolution des attentes et des objectifs de l’équipe pour atteindre le résultat actuel.~\\

Les premières difficultés sont de natures académiques et intellectuelles. En effet, nous avions pour objectif de réaliser notre propre radiogoniomètre à effet Doppler, couvrant le 2.4 GHz ambition réalisable mais complexe dans le temps imparti. La réalisation d’un tel système implique une solide maitrise en électronique et théorie des ondes, qualité que nous aurions pu acquérir avec plus de délais. De l’implémentation numérique de composants absents du logiciel Proteus, en passant par la réalisation du schéma électrique puis du circuit imprimé, nos notions en électroniques étaient trop sommaires pour se lancer dans une telle réalisation. Cela demeurait cependant la seule solution pour obtenir un système complet en fin d’année, le prix d’un tel radiogoniomètre avoisinant les 2000\euro sur le marché. 
La complexité de ce système a été accentuée par des retards de livraisons et de commande. Si certains de ces retards sont entièrement dus à notre innocence dans la conduite d’un projet, d’autres incombent uniquement aux fournisseurs. En effet, la commande de nos antennes accuse plus d’un mois de retard et devrait donc être annulé. Le dialogue avec le fournisseur n’y changeant rien, ce retard nous a d’autant plus conduit à faire évoluer nos attentes et nous recentrer sur la deuxième partie de notre projet, la gestion de la détection. D’une détection réelle, nous avons donc pris la décision de présenter une détection simulée, décision approuvée lors de la soutenance intermédiaire. 
~\\

Dans un second temps, les difficultés furent de natures humaines et relationnelles. Chaque groupe possède son élément moteur. Or, suite à des problèmes de santé, Michael RIGAUD a dû se retirer temporairement du groupe suite à sa période de convalescence. Cette période, démarrant en même temps que nos évolutions d’objectifs, a donc vu notre productivité et notre motivation diminuer à cause du retard accumulé sur la réalisation des goniomètres pour au final en abandonner la réalisation. Cette période aurait à l’inverse dû être un second départ pour l’ensemble du groupe afin de se relancer. 
Toutes ces difficultés rencontrées nous auront ouvert les yeux sur les obstacles à la réalisation d’un projet, obstacles auxquels nous pourront dorénavant nous préparer à l’avenir.


%%% Local Variables: 
%%% mode: latex
%%% TeX-master: "../rapport"
%%% End: 
